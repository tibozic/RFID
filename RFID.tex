\documentclass[]{article}

% packages
\usepackage{biblatex} %Imports biblatex package

% config
% Biblatex
\addbibresource{resources.bib} %Import the bibliography file
\nocite{*}

\title{RFID \\
       \large Radio Frequency Identification}
\date{}

\begin{document}

\maketitle

\newpage

\tableofcontents

\newpage

\section{Osnove}
RFID tehnologija omogo\v{c}a, da s pomo\v{c}jo
radiofrekven\v{c}nih polji zaznamo in prepoznamo objekte oziroma
predmete.

Tehnologija deluje tako, da imamo 2 napravi: bralca ter oznako.
Oznaka vedno \v{c}aka na bralca. Ko bralec oznako prebudi
mu ta odgovori s svojo identifikacjisko \v{s}tevilko in po
mo\v{z}nosti tudi drugimi podatki.

Poznamo ve\v{c} vrst oznak:
\begin{itemize}
  \item Pasivne oznake
  \item Delno pasivne oznake
  \item Aktivne oznake
\end{itemize}

\subsection{Pasivne oznake}
Te v celoti poganja bral\v{c}ev signal in ne potrebujejo svojega
napajanja.

\subsection{Delno pasivne oznake}
Te le delno poganja bralcev signal, za izvajanje dodatne logike
pa potrebujejo tudi svoje napajanje.

\subsection{Aktivne oznake}
Te za delovanje v celoti uporabljajo svoje napajanje.

\newpage

\section{Viri}
\printbibliography[heading=none] %Prints bibliography

\end{document}
