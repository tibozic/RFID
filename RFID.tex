\documentclass[]{article}

% packages
\usepackage{biblatex} % Imports biblatex package
\usepackage{graphicx}
\usepackage{float}
\usepackage{enumitem} % for advanced enumeration

% config
% Biblatex
\addbibresource{resources.bib} %Import the bibliography file
\nocite{*}

% image path
\graphicspath{ {./images/} }

\title{RFID \\
       \large Radio Frequency Identification}
\date{}

\begin{document}

\maketitle

\clearpage

\tableofcontents

\clearpage

\section{Zgodovina}
Leta 1945 je Léon Theremin za Sovjetsko zvezo izumil "the Thing",
prislu\v{s}kovalno napravo brez vgrajene baterije (pasivna).

To napravo so skrili v rezbarski ornament in ga podarili ameri\v{s}ki
ambasadi. V bli\v{z}njo stavbo so namestili radijski oddajnik, ki je
oddajal radijske valove proti napravi.

Zvo\v{c}ni valovi pogovorov v ameri\v{s}ki ambasadi so vibririali
diafragmo v napravi, kar je posledi\v{c}no povzro\v{c}ilo modulacijo
oddanega radijskega valova oddajne naprave in s tem prenos pogovorov
ruskim radijskim sprejemnikom. 

\v{C}eprav je bila ta naprava prikrita prislu\v{s}kovalna naprava in
ne identifikacijska zna\v{c}ka, se lahko \v{s}teje za predhodnico
RFID, ker so jo napajali in aktivirali valovi iz zunanjega vira.

Prvi pravi primer RFID tehnologije je bil predstavljen leta 1973.
Naprava je bila pasivna (napajali so jo zunanji radijski valovi) in je
vsebovala transponder ter 16 bitni spomin.

\begin{figure}[H] % H means exactly in place
  \centering
  \includegraphics[width=0.8\textwidth]{the_thing.jpg}
  \caption{The thing}
\end{figure}

\section{Osnove}
RFID je tehnologija, ki uporablja radijske valove za identifikacijo
in sledenje predmetov.

RFID sistem uporablja zna\v{c}ke (tags) pritrjene na predmete ter
radijski oddajnik/sprejemnik, imenovan \v{c}italec (reader), ki
po\v{s}lje signal zna\v{c}ki (jo aktivira) in prebere njen odgovor.
Ta odgovor lahko nato posreduje drugi napravi za nadaljno obdelavo. \\

\noindent
Sestavni deli zna\v{c}ke:
\begin{itemize}
  \item mikrokrmilnik (integrirano vezje, ki shranjuje in obdeluje
    informacije ter modulira in demodulira radiofrekven\v{c}ne (RF)
    signale)
  \item antena, ki se uporablja za sprejemanje in oddajanje radijskih
    signalov
  \item baterija (samo v aktivnih zna\v{c}kah)
  \item obstojni pomnilnik za shranjevanje podatkov
\end{itemize}

\subsection{Prednosti RFID}
RFID ima \v{s}tevilne prednosti pred tradicionalnimi sistemi
identifikacije in sledenja.
\begin{itemize}
  \item \v{c}italec lahko zna\v{c}ke bere na daljavo
  \item ni potrebno, da je zna\v{c}ka v vidnem polju \v{c}italca
  \item zna\v{c}ke lahko hranijo velike koli\v{c}ine informacij
    (ve\v{c} kB)
  \item omogo\v{c}a zelo hitro in natan\v{c}no branje ve\v{c}jih
    koli\v{c}in podatkov
  \item so odporne na zunanje dejavnike 
\end{itemize}

\subsection{Slabosti RFID}
\begin{itemize}
  \item ena glavnih je zasebnost, saj je lahko zna\v{c}ka prebrana
    brez privolitve nosilca
  \item RFID tehnologija potrebuje precej\v{s}njo investicijo v
    potrebno strojno in programsko opremo
\end{itemize}

\clearpage
\subsection{Delovanje}
RFID zna\v{c}ka se aktivira, ko pride v doseg RFID \v{c}italca:
\begin{enumerate}
  \item Čitalec odda radijski signal
  \item Antena značke ta signal sprejme
    \begin{enumerate}[label=\alph*.]
      \item Značka lahko ta signal uporabi za napajanje mikrokrmilnika (pasivna značka)
    \end{enumerate}
  \item Mikrokrmilnik značke pošlje podatke shranjene na znački preko antene kot odgovor (v obliki radijskega signala) čitalcu
\end{enumerate}

\noindent
RFID zna\v{c}ka sprejme sporo\v{c}ilo in se nato odzove s svojimi
podatki.

Te podatki so lahko edinstvena serijska \v{s}tevilka zna\v{c}ke ali
pa informacije povezane z izdelkom, kot so, datum proizvodnje itd\ldots

Ker imajo zna\v{c}ke unikatne serijske \v{s}tevilke, lahko RFID sistem
razlikuje med ve\v{c} zna\v{c}kami, ki so isto\v{c}asno v dosegu
\v{c}italnika in s tem bere podatki z vseh zna\v{c}k hkrati.

\subsection{Vrste sistemov/zna\v{c}k}
\subsubsection{Vrste zna\v{c}k glede na napajanje}
\begin{enumerate}
  \item Pasivne \\
    Za napajanje uporabljajo radijski signal \v{c}italca. So
    manj\v{s}e in cenej\v{s}e, a za aktivacijo potrebujejo
    pribli\v{z}no tiso\v{c}krat mo\v{c}nej\v{s}i signal.
  \item Delno pasivne \\
    Imajo vgrajeno baterijo in se aktivirajo, ko zaznajo prisotnost
    \v{c}italca.
  \item Aktivne \\
    Imajo vgrajeno baterijo in periodi\v{c}no oddajajo svoj ID preko
    radijskega signala.
\end{enumerate}

\subsection{Vrste sistemov glede na kombinacijo \v{c}italec-zna\v{c}ka}
\begin{itemize}
  \item Pasivni \v{c}italec aktivna zna\v{c}ka (PRAT)
    \begin{itemize}
      \item \v{c}italec samo sprejema radijo signale, ki jih
        zna\v{c}ka periodi\v{c}no oddaja
      \item uporablja se za varovanje in nadzor sredstev
    \end{itemize}
  \item Aktivni \v{c}italec pasivna zna\v{c}ka (ARPT)
  \item Aktivni \v{c}italec aktivna zna\v{c}ka (ARAT)
\end{itemize}

\noindent
Fiksni \v{c}italniki ustvarijo specifi\v{c}no obmo\v{c}je branja ki
ga je mogo\v{c}e strogo nadzorovati. To omogo\v{c}a visoko definirano
obmo\v{c}je branja, ko zna\v{c}ke vstopijo in izstopijo iz območja
branja. \\
Mobilni \v{c}italniki so lahko ro\v{c}ni ali name\v{s}\v{c}eni na
vozi\v{c}kih/vozilih.

\subsection{Vrste sistemov glede na frekvenco}
RFID lo\v{c}imo tudi glede na frekvenco signala:
\begin{itemize}
  \item Nizko frekvenčne
    \begin{itemize}
      \item Razpon: 120 – 150 kHz
      \item Doseg: 10 cm
      \item Prenosna hitrost: nizka
    \end{itemize}
  \item Visoko frekvenčne
    \begin{itemize}
      \item Razpon: 13.56 MHz
      \item Doseg: 0.1 – 1 m
      \item Prenosna hitrost: nizka/srednja
    \end{itemize}
  \item Ultra visoko frekvenčne
    \begin{itemize}
      \item Razpon: 865 – 928 MHz
      \item Doseg: 1 – 12 m
      \item Prenosna hitrost: srednja/visoka
    \end{itemize}
\end{itemize}

\subsection{Komunikacija med \v{c}italcem in zna\v{c}ko}
Zna\v{c}ka ima lahko lo\v{c}en oddajnik in sprejemnik. To pomeni, da
zna\v{c}ka lahko odgovori z signalom druga\v{c}ne frekvence, kot je
bil signal \v{c}italca.

EPC (Electronic Product Code) je trenutni standard, ki se uporablja
za enli\v{c}no identifikacijo zna\v{c}ke in produkta na globalnem
nivoju. \\

\noindent
Sestava EPC po bitih:
\begin{itemize}
  \item 8: glava, ki označuje verzijo protokola
  \item 28: identifikacijska \v{s}tevilka organizacije (dodeli EPC
    Global consortium), ki upravlja s podatki zna\v{c}ke
  \item 24: skupina produkta
  \item 36: serijska \v{s}tevilka zna\v{c}ke
\end{itemize}

\subsubsection{Kolizije}
Pogosto se ve\v{c} kot ena zna\v{c}ka odzove na \v{c}italec oznak.
Zaznavanje kolizij oz. trkov je pomembno za omogočanje branja
podatkov.

Za izolacijo dolo\v{c}ene oznake se uporabljata dva razli\v{c}na
protokola, ki omogo\v{c}ata branje podatkov zna\v{c}ke tudi, \v{c}e
se nahaja v bli\v{z}ini drugih zna\v{c}k. \\

\noindent
\textbf{Aloha s \v{c}asovnimi rezinami} \\
\v{C}asovno okno komunikacije razdelimo na enako velike \v{c}asovne
rezine (npr. 1 sekunda).

Naprave lahko pri\v{c}nejo oddajo podatkov le ob za\v{c}etku vsake
\v{c}asovne rezine. V primeru hkratnega oddajanja dveh ali ve\v{c}
naprav se zgodi kolizija in mora vsaka naprava po\v{c}akati
naklju\v{c}no \v{s}tevilo \v{c}asovnih rezin pred ponovno oddajo.

\v{C}italec po\v{s}lje inicializacijski signal in parameter.
Zna\v{c}ke s pomo\v{c}jo parametra izra\v{c}unajo naklju\v{c}no
\v{s}tevilo.  \\

\noindent
\textbf{Protokol prilagodljivega binarnega drevesa} \\
\v{C}italec po\v{s}lje inicializacijski signal in nato posreduje po
en bit ID zna\v{c}ke naenkrat. Odzivajo se le zna\v{c}ke z
ujemajo\v{c}imi biti. S\v{c}asoma se samo ena zna\v{c}ka ujema s
celotnim nizom ID-ja.

\begin{figure}[H] % H means exactly in place
  \centering
  \includegraphics[width=\textwidth]{binary_tree_collision.png}
  \caption{Primer izogibanja koliziji z binarnim drevesom}
\end{figure}

\subsection{Primeri uporabe}
\begin{itemize}
  \item omgo\v{c}ajo efektivno vodenje sredstev (zaloga, vrsta
    produkta)
  \item za\v{s}\v{c}ita proti kraji v trgovinah in skladi\v{s}\v{c}ih
  \item kontrola dostopa v zavarovanih obmo\v{c}jih
  \item identifikacija \v{z}ivali
  \item vodenje knji\v{z}nic
\end{itemize}

\clearpage
\subsection{Varnost}
Glavna varnostna skrb RFID tehnologije je nedovoljeno sledenje in
branje podatkov na zna\v{c}kah.

\noindent
Za varnost lahko poskrbimo na 2 na\v{c}ina:
\begin{enumerate}
  \item Uporabimo RFID, ki deluje na kraj\v{s}e razdalje za visoko
    rizi\v{c}ne dokumente (potni list, pla\v{c}ilne kartice). Z
    uporabo nizkofrekven\v{c}nih zna\v{c}k je verjetnost za
    ``skimming'' (nedovoljeno branje z zna\v{c}ke) manj\v{s}a. Kljub
    temu pa je bilo dokazano, da je \v{s}e vedno mo\v{z}no prebrati
    nizkofrekven\v{c}ne zna\v{c}ke tudi na razdaljah ve\v{c} metrov.
  \item Uporabimo kriptografski pristop:
    \begin{enumerate}[label=\alph*.]
        \item Cover-coding: Zna\v{c}ka generira naklju\v{c}no
          \v{s}tevilo. \v{C}italec uporabi to \v{s}tevilo za
          enkripcijo vseh nadaljnih sporo\v{c}il, ki jih po\v{s}lje
          zna\v{c}ki.
        \item Rolling code: \v{C}italec in zna\v{c}ka za vsako
          poslano sporo\v{c}ilo dogovorita za nov \v{s}ifririni
          klju\v{c}.
    \end{enumerate}
\end{enumerate}

\section{Protokol}
Iz standarda ISO/IEC 18000 (Information technology — Radio
frequency identification for item management), specifi\v{c}no 6.
del (komunikacija na UHF).

Za druge frekvence je standard podoben, z manj\v{s}o/ve\v{c}jo
zmogljivostjo. Poleg tega standarda, mora vsaka naprava
tudi slediti lokalnim standardom uporabljanja frekvenc. \\

\noindent
Standard podpira dve vrsti naprav:
\begin{itemize}
    \item Vrsta A - uporablja Pulse interval encoding (PIE) in
      ALOHA algoritem za prepre\v{c}itev kolizji (tem se bomo
      tukaj bolj posvetili)
    \item Vrsta B - uporablja Manchester encoding in binary tree
      collision za prepre\v{c}itev kolizji
\end{itemize}

\clearpage
\subsection{Komunikacija}
\subsubsection{PIE Kodiranje}
Pulse Interval Encoding (Impulzno Intervalno Kodiranje) deluje tako,
da izmeri \v{c}as med 2 negativnima frontama. Ta \v{c}as dolo\v{c}i
podatek ('1' ali '0').

\begin{figure}[H] % H means exactly in place
  \centering
  \includegraphics[width=\textwidth]{pie_graph.png}
  \caption{PIE - Dol\v{z}ina 1 Tari}
\end{figure}

\noindent
\emph{Tari} predstavlja \v{c}as med 2 negativnima frontama pri
vrednosti '0'. Za predstavitev '1' se pogosto uporablja signal
dol\v{z}ine 2 Tari. \\
SOF (Start of frame) je predstaviljen z 1 Tari + 3 Tari. \\
EOF (End of frame) je predstavljen z 4 Tari. \\

\begin{figure}[H] % H means exactly in place
  \centering
  \includegraphics[width=\textwidth]{pie_symbols.png}
  \caption{PIE - Simboli}
\end{figure}
 
Ko zna\v{c}ka bere te vrednosti, moramo imeti nekaj tolerance za
motnje pri prenosu. Vrednosti Tari na zna\v{c}ki se kalibrira pri
SOF.

\begin{figure}[H] % H means exactly in place
  \centering
  \includegraphics[width=\textwidth]{pie_tag_headroom.png}
  \caption{PIE - Zaznavanje vrednosti na zna\v{c}ki}
\end{figure}

Pred za\v{c}etkom po\v{s}iljanja mora \v{c}italec poskrbeti, da je
pretekel dolo\v{c}en \v{c}as ti\v{s}ine (300$\mu$s). \\
Po poslanem EOF, mora \v{c}italec ohraniti enakomern signal, da
je zna\v{c}ka napajana za njen odgovor.
 
\begin{figure}[H] % H means exactly in place
  \centering
  \includegraphics[width=\textwidth]{pie_command.png}
  \caption{PIE - Po\v{s}iljanje ukaza}
\end{figure}


\subsubsection{ALOHA}
Za prepre\v{c}evanje kolizji se uporablja ALOHA protokol. \\

\noindent
RFID uporablja protokol ALOHA, tako da zna\v{c}ke razdeli v kroge.
Ko prvi\v{c} pride do kolizije, morajo vse zna\v{c}ke, ki so
sodelovale v prej\v{s}njem krogu po\v{s}ilanja pognati svoj
naklju\v{c}ni generator \v{s}tevil. Vrednost, ki je bila generirana,
zna\v{c}ki pove v katerem krogu naj za\v{c}ne po\v{s}iljati podatke,
pri \v{c}emer vrednost 0 pomeni naslednji krog.

\v{C}e v naslednjem krogu ponovno pride do kolizije, morajo vse
zna\v{c}ke, ki niso sodelovale v po\v{s}iljanju svoj krog pove\v{c}ati
za 1. Vse zna\v{c}ke ki pa so sodelovale v po\v{s}iljanju ponovno
po\v{z}enejo svoj generator \v{s}tevil in dobijo nov krog. \\

\noindent
Vsaka zna\v{c}ka na za\v{c}etku komunikacije tudi generira svoj
podpis, ki se kasneje uporablja pri komunikaciji z \v{c}italnikom.
Zna\v{c}ka na za\v{c}etku komunikacje svoj podpis posreduje
\v{c}italniku, ta pa mora v vse naslednje pakete, ki so namenjeni tej
zna\v{c}ki vklju\v{c}iti njen podpis. Zna\v{c}ka vse pakete, ki ne
vsebujejo njenega podpisa ignorira, saj misli, da niso namenjeni njej.

\clearpage
\subsubsection{Ukazi in odgovori}
Protokol deluje, na principu \v{c}italec govori prvi. To pomeni,
da zna\v{c}ka ne odgovori, dokler ne prejme pravilno dekodiranega
ukaza. \\

\noindent
Zna\v{c}ka je v resnici kon\v{c}ni stroj:
\begin{figure}[H] % H means exactly in place
  \centering
  \includegraphics[width=\textwidth]{tag_state_machine.png}
  \caption{Kon\v{c}ni stroj zna\v{c}ke}
\end{figure}

\noindent
\textbf{RF Field off state} pomeni, da zna\v{c}ka ne dobiva
napajanja. \\
\textbf{Ready state} pomeni, da znacka dobiva dovolj napajanja za
pravilno delovanje. \\
\textbf{Quiet state} pomeni, da bo zna\v{c}ka odgovorila na vse ukaze
z njenim podpisom, ne bo pa odgovarjala na ukaze brez podpisov. \\
\textbf{Selected state} pomeni, da zna\v{c}ka odgovarja tudi na
ukaze, ki ne vsebujejo katerega koli podpisa. \\
\textbf{Round\_active state} pomeni, da zna\v{c}ka sodeluje v
naslednjem poskusu po\v{s}iljanja (ALOHA). \\
\textbf{Round\_standby state} pomeni, da zna\v{c}ka ne sodeluje v
naslednjem poskusu po\v{s}iljanja (ALOHA). \\

Ukazi in odgovori se izmenjujejo v obliki okvirjev. \\
Oblika ukaza:
\begin{figure}[H] % H means exactly in place
  \centering
  \includegraphics[width=\textwidth]{command_frame.png}
  \caption{Oblika okvirja pri ukazu}
\end{figure}

\noindent
\v{C}e ukaz vsebuje SUID zastavico, mora zna\v{c}ka primerjati svoj
podpis, s tistim, ki ga je poslal \v{c}italec in odgovori le v
primeru, da se ta ujema. \v{C}e ukaz ne vsebuje SUID zastavice, na
ta ukaz odgovorijo vse zna\v{c}ke, ki so v izbranem stanju.

\noindent
RFU - Reserved for future use \\

\noindent
Ukazi se delijo na 4 razli\v{c}ne skupine:
\begin{enumerate}
  \item Mandatory - obvezni ukazi
  \item Optional - neobvezni ukazi (ob nepodpiranju ukaza, zna\v{c}ka
    odgovori z napako ``command not supported'')
  \item Custom - ukazi, ki jih definira proizvajalec in so odprti
    za uporabnike
  \item Proprietary - ukazi, ki jih uporablja proizvajalec za
    testiranje, ni nujno da so odprte za uporabnike
\end{enumerate}

\noindent
\v{C}italec mora podpirati vse obvezne in neobvezne ukaze.

\noindent
Primeri ukazov:
\begin{itemize}
  \item Init\_round
  \item Next\_slot
  \item Close\_slot
  \item Standy\_round
  \item Read\_block
  \item Get\_system\_information
  \item Write\_block
  \item Write\_multiple\_blocks
  \item Lock\_blocks
  \item Get\_block\_lock\_status
\end{itemize}

\subsubsection{Ukazi za izbiranje skupin zna\v{c}k}
Za to se uporabljata ukaza \texttt{GROUP\_SELECT} in
\texttt{GROUP\_UNSELECT}. \\
Obema ukazoma dodamo tudi primerjalno pripono: EQ, NE, GT, LT. \\
Zna\v{c}ka, ki zadostuje pogoju odgovori s svojim UID. \\
Primerja se lahko katerakoli vrednost, za katero vemo na katerem
naslovu se nahaja. 

\subsubsection{Napake}
Oblika paketa z sporocilom o napaki:
\begin{figure}[H] % H means exactly in place
  \centering
  \includegraphics[width=\textwidth]{response_frame.png}
  \caption{Oblika okvirja pri odgovoru}
\end{figure}

Vrste napak:
\begin{figure}[H] % H means exactly in place
  \centering
  \includegraphics[width=\textwidth]{response_error_code.png}
  \caption{Vrste napak}
\end{figure}

\v{C}italec mora tudi poslati odgovor o uspe\v{s}no prejetem
sporo\v{c}ilu iz strani zna\v{c}ke.

\subsubsection{Manchester kodiranje}

V manchester kodirnaju se za predstavitev vrednosti uporabljajo
prehodi med stanji.

\begin{table}[H] % H means exactly in place
  \begin{center}
    \begin{tabular}{c | c}
      Vrednost & Prehod \\
      \hline
      1 & High to Low \\
      0 & Low to High \\
    \end{tabular}
  \end{center}
  \caption{Tabela vrednosti v Manchester kodiranju}
\end{table}

\begin{figure}[H] % H means exactly in place
  \centering
  \includegraphics[width=\textwidth]{manchester_encoding.png}
  \caption{Manchester kodiranje}
\end{figure}

\subsection{Shramba}
Protokol dovoljuje do 256 blokov podatkov, kjer je vsak blok
velikosti do 256 bitov (skupaj 64kBits). Kasnej\v{s}e iteracije
tega standarda dovoljujejo raz\v{s}iritev te shrambe.

Vsak blok podatkov je lahko zakljenjen uporabni\v{s}ko ali
tovarni\v{s}ko. Uporabni\v{s}ko zaklenjene bloke je mogo\v{c}e
odkleniti z uporabni\v{s}ko dolo\v{c}enim geslom, medtem ko
tovarni\v{s}ko zaklenjenih blokov ni mogo\v{c}e odkleniti.

\clearpage
\section{Zaklju\v{c}ek}
RFID je zmogljiva tehnologija, ki ima \v{s}tevilne primere uporabe v
razli\v{c}nih panogah. Ponuja \v{s}tevilne prednosti pred
tradicionalnimi sistemi identifikacije in sledenja, kot so branje na
daljavo, hiter in natan\v{c}en zajem podatkov ter odpornost. Vendar
pa ima tudi nekaj pomanjkljivosti, kot so pomisleki glede zasebnosti
in varnosti ter visoki stro\v{s}ki implementacije. Kot pri vsaki
tehnologiji je tudi pri RFID pomembno pretehtati prednosti in
slabosti ter oceniti, ali je prava re\v{s}itev za dolo\v{c}en primer
uporabe.

\clearpage

\section{Viri}
\printbibliography[heading=none] %Prints bibliography

\end{document}
