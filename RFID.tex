\documentclass[]{article}

% packages
\usepackage{biblatex} %Imports biblatex package

% config
% Biblatex
\addbibresource{resources.bib} %Import the bibliography file
\nocite{*}

\title{RFID \\
       \large Radio Frequency Identification}
\date{}

\begin{document}

\maketitle

\clearpage

\tableofcontents

\clearpage

\section{Osnove}
RFID tehnologija omogo\v{c}a, da s pomo\v{c}jo
radiofrekven\v{c}nih polji zaznamo in prepoznamo objekte oziroma
predmete.

Tehnologija deluje tako, da imamo 2 napravi: bralca ter oznako.
Oznaka vedno \v{c}aka na bralca. Ko bralec oznako prebudi
mu ta odgovori s svojo identifikacjisko \v{s}tevilko in po
mo\v{z}nosti tudi drugimi podatki.


\subsection{RFID v primerjavi z QR kodami}
Poleg drugih primerov uporabe, se RFID tehnologija uporablja
tudi za vodenje inventarja v skladi\v{s}\v{c}ih. \\

\noindent
Zakaj bi uporablja RFID namesto QR kod? \\
QR kode so v primerjavi z RFID omejene tehnolo\v{s}ko kot tudi
\v{c}love\v{s}ko:
\begin{itemize}
  \item QR kode potrebujejo kodo v vidnem polju
  \item Skenirati je potrebno eno po eno
\end{itemize}

\noindent
Medtem ko RFID tak\v{s}nih omejitev nimajo:
\begin{itemize}
    \item RFID uporablja radjisko frekven\v{c}ne valove, ki ne
      potrebujejo vidnega polja
    \item RFID bralnik lahko zazna ve\v{c} tiso\v{c} oznak v
      sekundi in jih vnese v ra\v{c}unalni\v{s}ki sistem
\end{itemize}

\section{Vrste oznak}
Poznamo ve\v{c} vrst oznak glede na napajanje, ki ga uporabljajo:
\begin{itemize}
  \item Pasivne oznake
  \item Delno pasivne oznake
  \item Aktivne oznake
\end{itemize}

RFID lahko lo\v{c}imo tudi po vi\v{s}inah frekvenc, ki jih
uporabljajo:
\begin{itemize}
  \item Nizko frekven\v{c}ne
  \item Visoko frekven\v{c}ne
  \item Ultra visko frekven\v{c}e
\end{itemize}


\subsection{Pasivne oznake}
Te v celoti poganja bral\v{c}ev signal in ne potrebujejo svojega
napajanja.


\subsection{Delno pasivne oznake}
Te le delno poganja bralcev signal, za izvajanje dodatne logike
pa potrebujejo tudi svoje napajanje.


\subsection{Aktivne oznake}
Te za delovanje v celoti uporabljajo svoje napajanje.

\clearpage

\section{Viri}
\printbibliography[heading=none] %Prints bibliography

\end{document}
